%%%%%%%%%%%%%%%%%%%%%%%%%%%%%%%%%%%%%%%%%
% kaobook
% LaTeX Template
% Version 1.3 (18/2/20)
%
% This template originates from:
% https://www.LaTeXTemplates.com
%
% For the latest template development version and to make contributions:
% https://github.com/fmarotta/kaobook
%
% Authors:
% Federico Marotta (federicomarotta@mail.com)
% Giuseppe Silano (g.silano89@gmail.com)
% Based on the doctoral thesis of Ken Arroyo Ohori (https://3d.bk.tudelft.nl/ken/en)
% and on the Tufte-LaTeX class.
% Modified for LaTeX Templates by Vel (vel@latextemplates.com)
%
% License:
% CC0 1.0 Universal (see included MANIFEST.md file)
%
%%%%%%%%%%%%%%%%%%%%%%%%%%%%%%%%%%%%%%%%%

%----------------------------------------------------------------------------------------
%	PACKAGES AND OTHER DOCUMENT CONFIGURATIONS
%----------------------------------------------------------------------------------------

\documentclass[
	fontsize=10pt, % Base font size
	twoside=false, % Use different layouts for even and odd pages (in particular, if twoside=true, the margin column will be always on the outside)
	%open=any, % If twoside=true, uncomment this to force new chapters to start on any page, not only on right (odd) pages
	%chapterprefix=true, % Uncomment to use the word "Chapter" before chapter numbers everywhere they appear
	%chapterentrydots=true, % Uncomment to output dots from the chapter name to the page number in the table of contents
	numbers=noenddot, % Comment to output dots after chapter numbers; the most common values for this option are: enddot, noenddot and auto (see the KOMAScript documentation for an in-depth explanation)
	%draft=true, % If uncommented, rulers will be added in the header and footer
	%overfullrule=true, % If uncommented, overly long lines will be marked by a black box; useful for correcting spacing problems
]{kaobook}

% Choose the language
\usepackage[spanish, mexico]{babel} % traduce a español
\usepackage[autostyle]{csquotes} % comillas mexicanas

% Load packages for testing
%\usepackage{blindtext}
%\usepackage{showframe} % Uncomment to show boxes around the text area, margin, header and footer
%\usepackage{showlabels} % Uncomment to output the content of \label commands to the document where they are used

% Load the bibliography package
\usepackage{styles/kaobiblio}
\addbibresource{bib.bib} % Bibliography file

% Load mathematical packages for theorems and related environments. NOTE: choose only one between 'mdftheorems' and 'plaintheorems'.
\usepackage{styles/mdftheorems}
%\usepackage{styles/plaintheorems}

\graphicspath{imgs/} % Paths in which to look for images

\makeindex[columns=3, title=Alphabetical Index, intoc] % Make LaTeX produce the files required to compile the index

\makeglossaries % Make LaTeX produce the files required to compile the glossary

\makenomenclature % Make LaTeX produce the files required to compile the nomenclature

%----------------------------------------------------------------------------------------

\begin{document}

%----------------------------------------------------------------------------------------
%	BOOK INFORMATION
%----------------------------------------------------------------------------------------

%\titlehead{Document Template}
% \subject{Subject}

\title[Template for the {\normalfont\texttt{kaobook}} Class]{Template for the {\normalfont\texttt{kaobook}} Class}
% \subtitle{Subtitle}

\author[JBG]{Johnny B. Goode}

\date{\today}

%\publishers{An Awesome Publisher}

%----------------------------------------------------------------------------------------

\frontmatter % Denotes the start of the pre-document content, uses roman numerals

%----------------------------------------------------------------------------------------
%	OPENING PAGE
%----------------------------------------------------------------------------------------

% \makeatletter
% \extratitle{
% 	% In the title page, the title is vspaced by 9.5\baselineskip
% 	\vspace*{9\baselineskip}
% 	\vspace*{\parskip}
% 	\begin{center}
% 		% In the title page, \huge is set after the komafont for title
% 		\usekomafont{title}\huge\@title
% 	\end{center}
% }
% \makeatother

%----------------------------------------------------------------------------------------
%	COPYRIGHT PAGE
%----------------------------------------------------------------------------------------

\makeatletter
%\uppertitleback{\@titlehead} % Header

\lowertitleback{
	\textbf{Disclaimer} \\
	You can edit this page to suit your needs. For instance, here we have a no copyright statement, a colophon and some other information. This page is based on the corresponding page of Ken Arroyo Ohori's thesis, with minimal changes.
	
	\medskip
	
	\textbf{No copyright} \\
	\cczero\ This book is released into the public domain using the CC0 code. To the extent possible under law, I waive all copyright and related or neighbouring rights to this work.
	
	To view a copy of the CC0 code, visit: \\\url{http://creativecommons.org/publicdomain/zero/1.0/}
	
	\medskip
	
	\textbf{Colophon} \\
	This document was typeset with the help of \href{https://sourceforge.net/projects/koma-script/}{\KOMAScript} and \href{https://www.latex-project.org/}{\LaTeX} using the \href{https://github.com/fmarotta/kaobook/}{kaobook} class.
	
	\medskip
	
	\textbf{Publisher} \\
	First printed in May 2019 by me%\@publishers
}
\makeatother

%----------------------------------------------------------------------------------------
%	DEDICATION
%----------------------------------------------------------------------------------------

\dedication{
	The harmony of the world is made manifest in Form and Number, and the heart and soul and all the poetry of Natural Philosophy are embodied in the concept of mathematical beauty.\\
	\flushright -- D'Arcy Wentworth Thompson
}

%----------------------------------------------------------------------------------------
%	OUTPUT TITLE PAGE AND PREVIOUS
%----------------------------------------------------------------------------------------

% Note that \maketitle outputs the pages before here

% If twoside=false, \uppertitleback and \lowertitleback are not printed
% To overcome this issue, we set twoside=semi just before printing the title pages, and set it back to false just after the title pages
\KOMAoptions{twoside=semi}
\maketitle
\KOMAoptions{twoside=false}

%----------------------------------------------------------------------------------------
%	PREFACE
%----------------------------------------------------------------------------------------

\chapter*{Preface}



%----------------------------------------------------------------------------------------
%	TABLE OF CONTENTS & LIST OF FIGURES/TABLES
%----------------------------------------------------------------------------------------

\begingroup % Local scope for the following commands

% Define the style for the TOC, LOF, and LOT
%\setstretch{1} % Uncomment to modify line spacing in the ToC
%\hypersetup{linkcolor=blue} % Uncomment to set the colour of links in the ToC
\setlength{\textheight}{23cm} % Manually adjust the height of the ToC pages

% Turn on compatibility mode for the etoc package
\etocstandarddisplaystyle % "toc display" as if etoc was not loaded
\etocstandardlines % "toc lines as if etoc was not loaded

\tableofcontents % Output the table of contents

\listoffigures % Output the list of figures

% Comment both of the following lines to have the LOF and the LOT on different pages
%\let\cleardoublepage\bigskip
%\let\clearpage\bigskip

\listoftables % Output the list of tables

\endgroup

%----------------------------------------------------------------------------------------
%	MAIN BODY
%----------------------------------------------------------------------------------------

\mainmatter % Denotes the start of the main document content, resets page numbering and uses arabic numbers
\setchapterstyle{kao} % Choose the default chapter heading style

\chapter{Introducción}
\section{Cristalografía de rayos X}
Actualmente la forma preponderante de obtener la estructura atómica de una macromolécula es a través de la cristalografía de rayos X (tabla \ref{tab:pdb-stats}). 

\begin{table}[h]
	\centering
	\begin{tabular}{@{}llr@{}}
		\toprule
		Método experimental & Total  & Porcentaje        \\ \midrule
		CRX     & 146963 & 88.84	\\
		RMN     & 13005  & 7.86		\\
		CME     & 5181   & 3.13		\\
		Varios  & 167    & 0.10		\\
		Otros 	& 106    & 0.06		\\
		Total   & 165422 & 100		\\ \bottomrule
	\end{tabular}%
	\caption[Número de estructuras depositadas por método experimental]{Número de estructuras depositadas en el PDB por método experimental. TODO: Falta explicar qué significa cada cosa.}
	\label{tab:pdb-stats}
\end{table}

De manera muy somera, el experimento de cristalografía de rayos X consiste en:

\begin{enumerate}
	\item Incidir rayos X sobre el cristal de la macromolécula de interés. 
	\item Obtener el patrón de difracción. 
	\item Rotar el cristal en cierto eje. 
	\item Repetir los pasos anteriores $n$ veces.
\end{enumerate}

Dos puntos que cabe resaltar son los siguientes: (\emph{i}) Los rayos X son difractados dentro del cristal macromolecular y dada la estructura repetitiva del mismo, se puede obtener una amplificación de este proceso puramente físico. Los rayos X difractados contienen información de la estructura macromolecular, por lo que es necesario detectarlos y mantener una copia digital de cada patrón de difracción para su posterior análisis. (\emph{ii}) En general, existen reglas de dedo para obtener un estimado útil de $n$ TODO: cita. 

Entre mayor información se obtenga, los pasos subsecuentes se vuelven menos complicados, por lo que sería fácil asumir que uno necesita exponer el cristal macromolecular cientos o miles de veces al haz de rayos X. Sin embargo, esto es raramente posible. El problema consiste en que los rayos X tienen la energía suficiente para ionizar la materia. En el caso de un cristal macromolecular, su estabilidad física se da por interacciones no covalentes, por lo que su desintegración no requiere de mucha energía. Además es evidente que al perderse el orden cristalino se pierde la amplificación del proceso de difracción y en consecuencia los patrones de difracción cada vez contienen menos información. Esto se conoce como daño por radiación y es una de las grandes limitantes de la cristalografía de rayos X.


\section{Crioprotección y desarrollo tecnológico de sincrotrones}

La primer estructura macromolecular determinada fue aquella de la mioglobina en 1958 \cite{Kendrew1958}. La forma de contender con el daño por radiación en aquellas épocas era utilizando decenas de cristales y promediar los patrones de difracción. Para 1966 es evidente que enfriar el cristal durante su exposición a los rayos X, ayuda a disminuir el daño por radiación \cite{Low1966}. La criocristalografía se desarrolla en los años siguientes y es hasta el año 2000 que se utiliza de manera rutinaria \cite{Garman2003}. Para entonces la noción general en el campo de la criocristalografía es que el daño por radiación era insignificante. Precisamente esta noción cambia en el mismo año, cuando tres estudios independientes muestran el efecto del daño por radiación en la entonces nueva generación de sincrotrones \cite{Teng2000, Ravelli2000, Weik2000}.

Una de las principales fuentes de rayos X es la radiación de sincrotrón. Una de las características de un sincrotrón es su brillo espectral\sidenote{Se define como la distribución del flujo de fotones en el espacio y el rango angular. El flujo se establece a su vez como el número de fotones por segundo que atraviesan un área definida por un ancho de banda dado \cite{Willmott2019}.}. La revolución tecnológica de los sincrotrones se nota en la diferencia del orden de magnitud del brillo espectral. Este aumento se ha permitido a pesar de que existe un incremento en el daño por radiación, porque da acceso a una gran ventaja: la posibilidad de utilizar cristales de menor tamaño\sidenote{La principal limitante de la cristalografía es obtener cristales macromoleculares de un tamaño adecuado: para una línea común esto significa al menos cien micrómetros en sus tres dimensiones, para una línea microfoco este valor puede disminuir un orden de magnitud.}.  Actualmente se está desarrollando la tecnología para cambiar la metodología de la colecta de datos, usando cristales macromoleculares nanométricos y con una fuente de rayos X más poderosa denominada XFEL (del inglés \emph{X-ray Free Electron Laser}) \cite{Martin-Garcia2016}. Existen ya varios estudios en los que se ha demostrado la posibilidad de obtener estructuras macromoleculares con esta nueva metodología \cite{Martin-Garcia2016}. Sin embargo, el acceso al tiempo experimental en un XFEL es actualmente muy limitado.

\section{Radioprotectores}
Al ser evidente que el daño por radiación aumentaba con el incremento en brillo, fue necesario buscar alternativas que ayudaran a mitigar el daño por radiación. Se han investigado alternativas pre y posteriores a la difracción con varios enfoques en los últimos 20 años\cite{Garman2017}. Una de las alternativas que resalta es el uso de moléculas pequeñas que interactuan con los radicales libres generados por la radiación. Estas moléculas se denominan radioprotectores.  Sin embargo, en la literatura científica existen varias incongruencias con respecto a la efectividad de los radioprotectores y es por esto que la comunidad cristalográfica no ha adoptado al cien por ciento el uso de radioprotectores de manera rutinaria \cite{Nowak2009, Allan2013}.




%\pagelayout{wide} % No margins
%\addpart{Title of the Part}
%\pagelayout{margin} % Restore margins

\chapter{Second Chapter}



\appendix % From here onwards, chapters are numbered with letters, as is the appendix convention

%\pagelayout{wide} % No margins
%\addpart{Appendix}
%\pagelayout{margin} % Restore margins

\chapter{Third Chapter}



%----------------------------------------------------------------------------------------

\backmatter % Denotes the end of the main document content
\setchapterstyle{plain} % Output plain chapters from this point onwards

%----------------------------------------------------------------------------------------
%	BIBLIOGRAPHY
%----------------------------------------------------------------------------------------

% The bibliography needs to be compiled with biber using your LaTeX editor, or on the command line with 'biber main' from the template directory

\defbibnote{bibnote}{Here are the references in citation order.\par\bigskip} % Prepend this text to the bibliography
\printbibliography[heading=bibintoc, title=Bibliography, prenote=bibnote] % Add the bibliography heading to the ToC, set the title of the bibliography and output the bibliography note

%----------------------------------------------------------------------------------------
%	NOMENCLATURE
%----------------------------------------------------------------------------------------

% The nomenclature needs to be compiled on the command line with 'makeindex main.nlo -s nomencl.ist -o main.nls' from the template directory

\nomenclature{$c$}{Speed of light in a vacuum inertial frame}
\nomenclature{$h$}{Planck constant}

\renewcommand{\nomname}{Notation} % Rename the default 'Nomenclature'
\renewcommand{\nompreamble}{The next list describes several symbols that will be later used within the body of the document.} % Prepend this text to the nomenclature

\printnomenclature % Output the nomenclature

%----------------------------------------------------------------------------------------
%	GREEK ALPHABET
% 	Originally from https://gitlab.com/jim.hefferon/linear-algebra
%----------------------------------------------------------------------------------------

\vspace{1cm}

{\usekomafont{chapter}Greek Letters with Pronounciation} \\[2ex]
\begin{center}
	\newcommand{\pronounced}[1]{\hspace*{.2em}\small\textit{#1}}
	\begin{tabular}{l l @{\hspace*{3em}} l l}
		\toprule
		Character & Name & Character & Name \\ 
		\midrule
		$\alpha$ & alpha \pronounced{AL-fuh} & $\nu$ & nu \pronounced{NEW} \\
		$\beta$ & beta \pronounced{BAY-tuh} & $\xi$, $\Xi$ & xi \pronounced{KSIGH} \\ 
		$\gamma$, $\Gamma$ & gamma \pronounced{GAM-muh} & o & omicron \pronounced{OM-uh-CRON} \\
		$\delta$, $\Delta$ & delta \pronounced{DEL-tuh} & $\pi$, $\Pi$ & pi \pronounced{PIE} \\
		$\epsilon$ & epsilon \pronounced{EP-suh-lon} & $\rho$ & rho \pronounced{ROW} \\
		$\zeta$ & zeta \pronounced{ZAY-tuh} & $\sigma$, $\Sigma$ & sigma \pronounced{SIG-muh} \\
		$\eta$ & eta \pronounced{AY-tuh} & $\tau$ & tau \pronounced{TOW (as in cow)} \\
		$\theta$, $\Theta$ & theta \pronounced{THAY-tuh} & $\upsilon$, $\Upsilon$ & upsilon \pronounced{OOP-suh-LON} \\
		$\iota$ & iota \pronounced{eye-OH-tuh} & $\phi$, $\Phi$ & phi \pronounced{FEE, or FI (as in hi)} \\
		$\kappa$ & kappa \pronounced{KAP-uh} & $\chi$ & chi \pronounced{KI (as in hi)} \\
		$\lambda$, $\Lambda$ & lambda \pronounced{LAM-duh} & $\psi$, $\Psi$ & psi \pronounced{SIGH, or PSIGH} \\
		$\mu$ & mu \pronounced{MEW} & $\omega$, $\Omega$ & omega \pronounced{oh-MAY-guh} \\
		\bottomrule
	\end{tabular} \\[1.5ex]
	Capitals shown are the ones that differ from Roman capitals.
\end{center}

%----------------------------------------------------------------------------------------
%	GLOSSARY
%----------------------------------------------------------------------------------------

% The glossary needs to be compiled on the command line with 'makeglossaries main' from the template directory

\newglossaryentry{computer}{
	name=computer,
	description={is a programmable machine that receives input, stores and manipulates data, and provides output in a useful format}
}

% Glossary entries (used in text with e.g. \acrfull{fpsLabel} or \acrshort{fpsLabel})
\newacronym[longplural={Frames per Second}]{fpsLabel}{FPS}{Frame per Second}
\newacronym[longplural={Tables of Contents}]{tocLabel}{TOC}{Table of Contents}

\setglossarystyle{listgroup} % Set the style of the glossary (see https://en.wikibooks.org/wiki/LaTeX/Glossary for a reference)
\printglossary[title=Special Terms, toctitle=List of Terms] % Output the glossary, 'title' is the chapter heading for the glossary, toctitle is the table of contents heading

%----------------------------------------------------------------------------------------
%	INDEX
%----------------------------------------------------------------------------------------

% The index needs to be compiled on the command line with 'makeindex main' from the template directory

\printindex % Output the index

%----------------------------------------------------------------------------------------
%	BACK COVER
%----------------------------------------------------------------------------------------

% If you have a PDF/image file that you want to use as a back cover, uncomment the following lines

%\clearpage
%\thispagestyle{empty}
%\null%
%\clearpage
%\includepdf{cover-back.pdf}

%----------------------------------------------------------------------------------------

\end{document}
