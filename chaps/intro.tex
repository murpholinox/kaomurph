\setchapterpreamble[u]{\margintoc}
\chapter{Introducción}
\labch{intro}
\section{Cristalografía de rayos X}
\labsec{crx}
Actualmente la forma preponderante de obtener la estructura atómica de una macromolécula es a través de la cristalografía de rayos X (tabla \ref{tab:pdb-stats}). 

\begin{table}[h]
	\centering
	\begin{tabular}{@{}llr@{}}
		\toprule
		Método experimental & Total  & Porcentaje        \\ \midrule
		CRX     & 146963 & 88.84	\\
		RMN     & 13005  & 7.86		\\
		CME     & 5181   & 3.13		\\
		Varios  & 167    & 0.10		\\
		Otros 	& 106    & 0.06		\\
		Total   & 165422 & 100		\\ \bottomrule
	\end{tabular}%
	\caption[Número de estructuras depositadas por método experimental]{Número de estructuras depositadas en el PDB por método experimental. TODO: Falta explicar qué significa cada cosa.}
	\labtab{tab:pdb-stats}
\end{table}

De manera muy somera, el experimento de cristalografía de rayos X consiste en:

\begin{enumerate}
	\item Incidir rayos X sobre el cristal de la macromolécula de interés. 
	\item Obtener el patrón de difracción. 
	\item Rotar el cristal en cierto eje. 
	\item Repetir los pasos anteriores $n$ veces.
\end{enumerate}

Dos puntos que cabe resaltar son los siguientes: (\emph{i}) Los rayos X son difractados dentro del cristal macromolecular y dada la estructura repetitiva del mismo, se puede obtener una amplificación de este proceso puramente físico. Los rayos X difractados contienen información de la estructura macromolecular, por lo que es necesario detectarlos y mantener una copia digital de cada patrón de difracción para su posterior análisis. (\emph{ii}) En general, existen reglas de dedo para obtener un estimado útil de $n$ TODO: cita. 

\section{Daño por radiación}
\labsec{dpr}
Entre mayor información se obtenga, los pasos subsecuentes se vuelven menos complicados, por lo que sería fácil asumir que uno necesita exponer el cristal macromolecular cientos o miles de veces al haz de rayos X. Sin embargo, esto es raramente posible. El problema consiste en que los rayos X tienen la energía suficiente para ionizar la materia. En el caso de un cristal macromolecular, su estabilidad física se da por interacciones no covalentes, por lo que su desintegración no requiere de mucha energía. Además es evidente que al perderse el orden cristalino se pierde la amplificación del proceso de difracción y en consecuencia los patrones de difracción cada vez contienen menos información. Esto se conoce como daño por radiación y es una de las grandes limitantes de la cristalografía de rayos X.

\section{Crioprotección}
\labsec{crio}
La primer estructura macromolecular determinada fue aquella de la mioglobina en 1958 \cite{Kendrew1958}. La forma de contender con el daño por radiación en aquellas épocas era utilizando decenas de cristales y promediar los patrones de difracción. Para 1966 es evidente que enfriar el cristal durante su exposición a los rayos X, ayuda a disminuir el daño por radiación \cite{Low1966}. La criocristalografía se desarrolla en los años siguientes y es hasta el año 2000 que se utiliza de manera rutinaria \cite{Garman2003}. Para entonces la noción general en el campo de la criocristalografía es que el daño por radiación era insignificante. Precisamente esta noción cambia en el mismo año, cuando tres estudios independientes muestran el efecto del daño por radiación en la entonces nueva generación de sincrotrones \cite{Teng2000, Ravelli2000, Weik2000}.

\section{Sincrotrones}
\labsec{sincro}
Una de las principales fuentes de rayos X es la radiación de sincrotrón. Una de las características de un sincrotrón es su brillo espectral\sidenote{Se define como la distribución del flujo de fotones en el espacio y el rango angular. El flujo se establece a su vez como el número de fotones por segundo que atraviesan un área definida por un ancho de banda dado \cite{Willmott2019}.}. La revolución tecnológica de los sincrotrones se nota en la diferencia del orden de magnitud del brillo espectral. Este aumento se ha permitido a pesar de que existe un incremento en el daño por radiación, porque da acceso a una gran ventaja: la posibilidad de utilizar cristales de menor tamaño\sidenote{La principal limitante de la cristalografía es obtener cristales macromoleculares de un tamaño adecuado: para una línea común esto significa al menos cien micrómetros en sus tres dimensiones, para una línea microfoco este valor puede disminuir un orden de magnitud.}.  Actualmente se está desarrollando la tecnología para cambiar la metodología de la colecta de datos, usando cristales macromoleculares nanométricos y con una fuente de rayos X más poderosa denominada XFEL (del inglés \emph{X-ray Free Electron Laser}) \cite{Martin-Garcia2016}. Existen ya varios estudios en los que se ha demostrado la posibilidad de obtener estructuras macromoleculares con esta nueva metodología \cite{Martin-Garcia2016}. Sin embargo, el acceso al tiempo experimental en un XFEL es actualmente muy limitado.

\section{Radioprotectores}
\labsec{radio}
Al ser evidente que el daño por radiación aumentaba con el incremento en brillo, fue necesario buscar alternativas que ayudaran a mitigar el daño por radiación. Se han investigado alternativas pre y posteriores a la difracción con varios enfoques en los últimos 20 años\cite{Garman2017}. Una de las alternativas que resalta es el uso de moléculas pequeñas que interactuan con los radicales libres generados por la radiación. Estas moléculas se denominan radioprotectores.  Sin embargo, en la literatura científica existen varias incongruencias con respecto a la efectividad de los radioprotectores y es por esto que la comunidad cristalográfica no ha adoptado al cien por ciento el uso de radioprotectores de manera rutinaria \cite{Nowak2009, Allan2013}.
